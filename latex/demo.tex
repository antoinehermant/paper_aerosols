\documentclass{article}
\usepackage{amsmath} % for additional math functionality
\begin{document}

Numerical demo:
\\\\
Let's take 2 absorbing surfaces with arbitrary overlying aerosol forcings:
\\\\
For $\alpha_1 = 0.2 \rightarrow F_1 = -10$ W.m $^{-2}$ 
\\
For $\alpha_2 = 0.1 \rightarrow F_2 = 2*F_1 = -20$ W.m $^{-2}$
\\
$$F_1*\alpha_1 = -10*0.2 = -2$$
$$F_2*\alpha_2 = -20*0.1 = -2$$
\\
In this example, surface 2 is twice as absorbing as surface 1. The induced aerosol forcing above surface 2 should be twice as great as above surface 1. We see that if we multiply the aerosol forcing by the surface albedo for both surfaces, we get the same value.
\\ 
This works if aerosol forcing directly proportional to surface albedo (which is suggested in my plots)
\\
Another way to see it is that the forcing is anti proportional to albedo, so we should normalise by the inverse of the albedo (which is same as multiplying):
$$\rightarrow \frac{F}{1/\alpha} = F*\alpha$$
\\\\
For positive forcing:
$$\rightarrow \frac{F}{1/(1-\alpha)} = F*(1-\alpha)$$

But:
\\
$$\frac{F_1}{(1 - \alpha_1)} = \frac{-20}{0.8} = -20*1.1 = 22$$
$$\frac{F_2}{(1 - \alpha_2)} = \frac{-10}{0.9} = -10*1.25 = 12.5$$
\\
Not linear, not the same as normalising because albedo = ]0;1[
\\\\
Thus I do think we have to multiply by the surface albedo, but it gives annoying units. Maybe we can think about a different plot that better shows the relationship. \\\\
For reflective surfaces (or induced positive aerosol forcing):
\\\\
For $\alpha_3 = 0.8 \rightarrow F_3 = 10$ W.m $^{-2}$ 
\\
For $\alpha_4 = 0.9 \rightarrow F_4 = 20$ W.m $^{-2}$
\\
$$F_3*\alpha_3 = -20*0.1 = -2$$
$$F_4*\alpha_4 = -10*0.2 = -2$$
\\
In this example, surface 1 is twice as absorbing as surface 2. The induced aerosol forcing above surface 1 should be twice as great as above surface 2. We see that if we multiply the aerosol forcing by the surface albedo for both surfaces, we get the same value.
\\ 
This works if aerosol forcing directly proportional to surface albedo (which is suggested in my plots)
\\\\
But:
\\
\end{document}
